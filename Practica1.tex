\documentclass[fleqn, 10pt]{article}

% Paquetes necesarios
\usepackage[utf8]{inputenc}
\usepackage[spanish]{babel}
\usepackage{amsthm, amsmath}
\usepackage{nccmath} %Para centrar ecuaciones
\usepackage{graphicx}
\usepackage{enumitem}

% Personalizo mi alfabeto
\DeclareMathAlphabet{\pazocal}{OMS}{zplm}{m}{n}
\newcommand{\Lb}{\pazocal{L}}

% Definimos los entornos para definiciones, teoremas, etc...
\theoremstyle{plain}
\newtheorem{proposicion}{Proposición}

\theoremstyle{definition}
\newtheorem{definition}{Definición}[section]
\newtheorem{example}{Ejemplo}[section]

%Definimos el título
\title{Teoría de Autómatas y Lenguajes Formales\\[.4\baselineskip]Práctica 1: ejercicio 1 }
\author{Irene, Recio López}
\date{\today}

%Comienzo del documento
\begin{document}

%Generamos el título
\maketitle

\section*{Calcula R$^3$ dado R = \{(1,1),(1,2),(2,3),(3,4)\}} 

Para R = \{(1,1),(1,2),(2,3),(3,4)\}

Para R$^2$ = \{(1,1),(1,2),(1,3),(2,4)\}

Para R$^3$ = \{(1,1),(1,2),(1,3),(1,4)\}

Podemos representar la solución de forma matricial

\begin{equation}
\begin{pmatrix}
1 & 0 & 0 & 0\\
1 & 0 & 0 & 0\\
0 & 1 & 0 & 0\\
0 & 0 & 1 & 0\\
\end{pmatrix}
\times
\begin{pmatrix}
1 & 0 & 0 & 0\\
1 & 0 & 0 & 0\\
1 & 0 & 0 & 0\\
0 & 1 & 0 & 0\\
\end{pmatrix}
=
\begin{pmatrix}
1 & 0 & 0 & 0\\
1 & 0 & 0 & 0\\
1 & 0 & 0 & 0\\
1 & 0 & 0 & 0\\
\end{pmatrix}
\end{equation} 
\section*{Comprobación mediante Octave} 
powerrelation({['1', '1'], ['1', '2'], ['2', '3'], ['3', '4']}, 3)

ans =

\{

  [1,1] = 11
  
  [1,2] = 12
  
  [1,3] = 13
  
  [1,4] = 14
  
\}
\end{document}




